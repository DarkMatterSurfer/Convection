
\documentclass[10pt]{article}
\linespread{1.25}

%%Make Parenthesis scale to fit whats inside
\newcommand{\parry}[1]{\left( #1 \right)}

%% Language and font encodings
\usepackage[english]{babel}
\usepackage[utf8x]{inputenc}
\usepackage[T1]{fontenc}
\usepackage{subcaption}
\usepackage[section]{placeins}

%% Sets page size and margins
\usepackage[a4paper,top=3cm,bottom=2cm,left=3cm,right=3cm,marginparwidth=1.75cm]{geometry}

%% Useful packages
\usepackage{amsmath}
\usepackage{amssymb}
\usepackage{amsfonts}
\usepackage{mathtools}
\usepackage{graphicx}
\usepackage{xcolor}
\usepackage[colorinlistoftodos]{todonotes}
\usepackage[colorlinks=true, allcolors=blue]{hyperref}
\usepackage{enumerate}
\usepackage{enumitem}
\usepackage{siunitx}
\usepackage{float}
\usepackage{scrextend}
\usepackage[final]{pdfpages}
\usepackage{pythonhighlight}

%%Header & Footer
\usepackage[myheadings]{fullpage}
\usepackage{fancyhdr}
\usepackage{lastpage}
\usepackage{graphicx, wrapfig, subcaption, setspace, booktabs}

%% Define \therefore command
\def\therefore{\boldsymbol{\text{ }
\leavevmode
\lower0.4ex\hbox{$\cdot$}
\kern-.5em\raise0.7ex\hbox{$\cdot$}
\kern-0.55em\lower0.4ex\hbox{$\cdot$}
\thinspace\text{ }}}

%% Units
\DeclareSIUnit\year{yr}
\DeclareSIUnit\dollar{\$}
\DeclareSIUnit\celcius{C^{\circ}}
\DeclareSIUnit\mole{mole}
\def\conclusion{\quad \Rightarrow \quad}

\newcommand{\angled}[1]{\left\langle #1 \right\rangle}
\newcommand{\pd}[1]{\partial_{#1}}
\renewcommand{\vec}[1]{\boldsymbol{#1}}
\newcommand{\M}[1]{\mathbf{#1}}
\newcommand{\grad}{\vec{\nabla}}
\newcommand{\cross}{\vec{\times}}
\newcommand{\laplacian}{\nabla^2}
\makeatletter
\DeclareRobustCommand{\pder}[1]{%
  \@ifnextchar\bgroup{\@pder{#1}}{\@pder{}{#1}}}
\newcommand{\@pder}[2]{\frac{\partial#1}{\partial#2}}
\makeatother
\begin{document}


%----------------------------------------------------------------------------------------
%	TITLE PAGE
%----------------------------------------------------------------------------------------

%----------------------------------------------------------------------------------------
% HEADER AND FOOTER
%----------------------------------------------------------------------------------------
\pagestyle{fancy}
\fancyhf{}
\setlength\headheight{12pt}
\fancyhead[L]{\textbf{Solar Convection Notes}}
\fancyhead[R]{\textbf{Liam O'Connor}}
\fancyfoot[R]{Page \thepage\ of \pageref{LastPage}}

\section{Anelastic Convection}

\begin{align*}
  \frac{\grad s}{c_p} = 0 &= \frac{1}{\gamma} \grad \log T_0 - \frac{\gamma - 1}{\gamma}\grad \log \rho_0 \\
  \frac{g}{T_0} &= \frac{R}{\mu}\left( \grad \log \rho_0 + \grad \log T_0 \right) \\
  \frac{\mu}{R}\frac{g}{T_0} &= \grad \log \rho_0 + \grad \log T_0 \\
  \grad \log \rho_0 &= \frac{\mu}{R}\frac{g}{T_0} - \grad \log T_0  \\
  0 &= \frac{1}{\gamma} \grad \log T_0 - \frac{\gamma - 1}{\gamma} \left( \frac{\mu}{R}\frac{g}{T_0} - \grad \log T_0 \right) \\
  0 &= \grad \log T_0 - (\gamma - 1) \frac{\mu}{R}\frac{g}{T_0} + (\gamma - 1)\grad \log T_0 \\
  (\gamma - 1) \frac{\mu}{R}\frac{g}{T_0} &= \grad \log T_0 + (\gamma - 1)\grad \log T_0 \\
  &= \grad \log T_0 + \grad (\log T_0^{\gamma - 1}) \\
  &= \grad (\log T_0^1T_0^{\gamma - 1}) \\
  q &\equiv \frac{\gamma - 1}{\gamma} \frac{\mu g}{R} \\
  \frac{q}{T_0(z)} &= \partial_z (\log T_0(z)) \\
  \frac{q}{T_0(z)} &= \frac{T_0'(z)}{T_0(z)} \\
  T_0(z) &= q z + T_b\\
  T_0'(z) &= q \\
  % T_0'(z) &= \frac{\gamma - 1}{\gamma} \frac{\mu g}{R} z + T_b\\
  % \int \frac{q}{T_0(z)} dz &= \int \partial_z (\log T_0(z)) dz \\
  % &= \log T_0(z) dz \\
\end{align*}
\begin{align*}
    0 &= \partial_z \log T_0 - (\gamma - 1) \partial_z \log \rho_0 \\
  &= \frac{T'_0(z)}{T_0(z)} - (\gamma - 1) \frac{\rho'_0(z)}{\rho_0(z)} \\
  &= \frac{q}{qz + T_b} - (\gamma - 1) \frac{\rho'_0(z)}{\rho_0(z)} \\
  &= \frac{1}{\gamma - 1}\frac{q}{qz + T_b} - \frac{\rho'_0(z)}{\rho_0(z)} \\
  &= \frac{1}{\gamma2 - 1}\frac{q}{qz + T_b} - \frac{\rho'_0(z)}{\rho_0(z)} \\
  \frac{d\rho_0}{dz} &= \frac{\rho_0(z)}{\gamma - 1}\frac{q}{qz + T_b} \\
  d\rho_0 &= \frac{\rho_0(z)}{\gamma - 1}\frac{qdz}{qz + T_b} \\
  (\gamma - 1) \frac{d\rho_0}{\rho_0}  &= \frac{qdz}{qz + T_b} \\
  (\gamma - 1) \int \frac{d\rho_0}{\rho_0}  &= \int \frac{qdz}{qz + T_b}\\
  (\gamma - 1) \log\rho_0  &= \log (qz + T_b)  + const. \\
  \log\rho_0  &= \frac{1}{(\gamma - 1)} \log (qz + T_b)  + const. \\
  \log\rho_0  &=  \log (qz + T_b)^{\frac{1}{(\gamma - 1)}}  + const. \\
  \rho_0  &=  \exp\left( \log (qz + T_b)^{\frac{1}{(\gamma - 1)}} + const.  \right) \\
  \rho_0  &=  \exp\left( \log (qz + T_b)^{\frac{1}{(\gamma - 1)}}\right)     \\
  \intertext{}
  \rho_0(z) &= (qz + 1)^{\frac{1}{(\gamma - 1)}}\\
  T_0(z) &= q z + 1\\
  % \rho_0(0) &= A T_b^{\frac{1}{(\gamma - 1)}}\\
  % \int \rho'_0(z) dz &= \int \frac{\rho_0(z)}{\gamma - 1}\frac{q}{qz + T_b} dz \\
\end{align*}























% Omega = r = uth / r
% uth = r ^2
% \begin{align*}
% \intertext{Inserting this expression for velocity into the $\theta$-component of the momentum equation gives}
%   \partial_t v + \nu\left( r^{-2}v - r^{-1}\partial_rv  - \partial_r^2v - \partial_z^2v \right) = r^{-1}\left( v\partial_z\psi - b\partial_za \right) + \partial_zb\partial_ra - \partial_za\partial_rb + \partial_z\psi\partial_rv - \partial_zv\partial_r\psi \\
%   \partial_t v + \nu\left( r^{-2}v - r^{-1}\partial_rv  - \partial_r^2v - \partial_z^2v \right) = r^{-1}\left( v\partial_z\psi - b\partial_za \right) + \partial_zb\partial_ra - \partial_za\partial_rb + \partial_z\psi\partial_rv - \partial_zv\partial_r\psi \\
%   \intertext{Then, taking the $\theta$-component of the curl of the vector momentum equation gives the following equation for $\grad^2\psi$}
%   \partial_t\grad^2\psi + \nu \left(\partial_z^2\psi/r^2-\partial_z^4\psi-\partial_r\partial_z^2\psi/r+\partial_r^2\psi/r^2-2 \partial_r^2\partial_z^2\psi - \partial_r^3\psi/r-\partial_r^4\psi \right)\quad\quad \\
%   -((2 b \partial_zb)/r)+(2 v \partial_z v)/r-\partial_z\psi \partial_r\partial_z^2\psi - \partial_ra \partial_z^3a+ \partial_ra \partial_r^2\partial_za + \partial_r\psi \partial_z^3\psi + \partial_r\psi \partial_r^2\partial_z\psi + \partial_z a (\partial_r\partial_z^2a+\partial_r^3a)-\partial_z\psi \partial_r^3\psi \\ 
%   % \grad^2f(r,z) &= r^{-1}\partial_r (r\partial_r f) + \partial_z^2f\\
%   % &= r^{-1}\partial_r f + \partial_r^2 f + \partial_z^2f\\
%   % \intertext{implies}
%   % \grad^2f(r,z) &= \grad^2\grad^2f(r,z) \\ 
%   % &= r^{-1}\partial_r (r^{-1}\partial_r f + \partial_r^2 f + \partial_z^2f)\\
%   % &\quad + \partial_r^2 (r^{-1}\partial_r f + \partial_r^2 f + \partial_z^2f)\\
%   % &\quad + \partial_z^2 (r^{-1}\partial_r f + \partial_r^2 f + \partial_z^2f)\\
%   % &= r^{-2}\partial_r^2 f - r^{-3}\partial_r f + r^{-1}\partial_r^3 f + r^{-1}\partial_r\partial_z^2f \\
%   % &\quad + r^{-1}\partial_r^3 f - r^{-2}\partial_r^2 f - r^{-2}\partial_r^2 f + 2r^{-3}\partial_r f + \partial_r^4 f + \partial_r^2\partial_z^2f\\
%   % &\quad + r^{-1}\partial_r\partial_z^2 f + \partial_r^2\partial_z^2 f + \partial_z^4f\\
% \end{align*}

\newpage
\section*{Modified Boussinesq}
\begin{align*}
  \grad\cdot\vec{u} &= 0 \\
  \partial_t\vec{u} + \vec{u}\cdot\grad\vec{u} &= -\rho_0^{-1}\grad p + \frac{\rho_1}{\rho_0}\vec{g} + \nu\grad^2\vec{u} \\
  \partial_t T + \vec{u}\cdot\grad T + w\grad_{\rm{ad}} + \grad\cdot[-k\grad \overline{T}] &= \chi\grad^2 T' + Q \\
  \frac{\rho_1}{\rho_0} &= -|\alpha|T
  \intertext{where $k$ is height-dependent and $\chi$ is not. The density $\rho$ is decomposed into a uniform, constant background $\rho_0$ and fluctuations $\rho_1$. The coefficient of thermal expansion $\alpha = \partial_T [\log\rho]$. The overbars denote horizontal averages and fluctuations with apostrophes. At the statistically-steady state, the heat fluxes must balance such that}
  \overline{F_{\rm{tot}}} &= \overline{F_{\rm{rad}}} + \overline{F_{\rm{conv}}} = \int Q dz + F_{\rm{bot}}
  \intertext{where $F_{\rm{bot}}$ is the flux carried at the bottom of the domain. $\overline{F_{\rm{tot}}}$ carries the total (horizontally-averaged) heat flux which is not constant in $z$ due to non-uniform heating in the domain $Q(z)$. The radiative flux is fully-defined from}
  \overline{F_{\rm{rad}}} &= -k\grad\overline{T}
  \intertext{making the convective flux}
  \overline{F_{\rm{conv}}} &= \overline{F_{\rm{tot}}} - \overline{F_{\rm{rad}}}
  \intertext{The temperature gradient}
  \nabla &\equiv -\partial_z \overline{T}
  \intertext{and radiative temperature gradient}
  \nabla_{\rm{rad}} &\equiv k^{-1}\overline{F_{\rm{tot}}}.
  \intertext{note that these $\nabla$'s are defined to be positive quantities. Marginal stability is achieved when}
  \nabla &= \nabla_{\rm{ad}}
  \intertext{which we assume to be constant. The classical Schwarzschild boundary of the convection zone is the height $z=L_s$ at which $\nabla_{\rm{rad}} = \nabla_{\rm{ad}}$ and $\overline{F_{\rm{conv}}}=0$.}
\end{align*}

According to the Buckingham $\pi$ theorem, there are nine fundamental input parameters in the equations, namely: $\rho_0, \alpha g, L_s, \nu, \chi, Q, \nabla_{ad}, k_{CZ}, k_{RZ}$
along with four fundamental dimensions: mass, length, time, and temperature.
Thus we have five independent prognostic parameters in setting up our system. For two of these parameters, we will choose the freefall Reynolds number and the Prandtl number, which are analogous to the Rayleigh and Prandtl numbers in RBC.
The remaining three parameters are $\mathcal{S}$ (stiffness), $\mathcal{P}$ (penetration parameter), and $\mu$ which sets the ratio between $\nabla_{\rm{rad}}$ and $\nabla_{\rm{ad}}$ in the convection zone (held constant).

After nondimensionalizing the original equations
\begin{align*}
  \grad\cdot\vec{u} &= 0 \\
  \partial_t \vec{u} + \vec{u}\cdot\grad\vec{u} &= -\grad\varpi + T\vec{\hat{z}} + \mathcal{R}^{-1}\grad^2\vec{u} \\
  \partial_t T + \vec{u}\cdot\grad T + w\nabla_{\rm{ad}} + \grad\cdot[-k\grad\overline{T}] &= (\rm{Pr}\mathcal{R}^{-1})\grad^2 T' + Q
\end{align*}
We further decompose the temperature into a time-stationary initial background profile and fluctuations
\begin{align*}
  T(x,y,z,t) &= T_0(z) + T_1(x,y,z,t).
\end{align*}
$T_0$ is constructed with $\nabla=\nabla_{\rm{ad}}$ for $z\leq L_s$ and $\nabla=\nabla_{\rm{rad}}$ for $z > L_s$.
The flux due to heating $F_H = Q_{mag}\Delta_H$ is simply the nonzero contribution of $Q$ times the region's height. 
The flux is specified by $0 < \mu = F_{bot}/F_H \ll 1$.
The stiffness $\mathcal{S}$ is the ratio of the bouyancy frequency $N^2$ to the characteristic convective frequency $f_{conv}^2 = 1$. Thus $\mathcal{S} = \nabla_{ad} - \nabla_{rad}$.


The radiative conductivity
\begin{align*}
  k_{rad} &= \frac{\chi}{\rho c_p}
  \intertext{where}
  \chi &= \frac{16\sigma T^3}{3\kappa \rho}
\end{align*}


\end{document}
